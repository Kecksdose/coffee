\documentclass[
    20pt
]{article}

\usepackage{polyglossia}
\setmainlanguage{german}

\usepackage{indentfirst}
\parindent 0pt

\usepackage[
    a4paper,
    left=20mm,
    right=20mm,
    top=20mm,
    bottom=20mm,
]{geometry}

\begin{document}
    \pagenumbering{gobble}
    \section*{Kaffeedienst {{ services[-1].start_date.strftime('%B %Y') }}}
    {{ foo }}
    Bitte täglich spülen und Uhrzeit eintragen. Einmal pro Woche reinigen,
    einmal pro Monat entkalken und je ein Kreuzchen in der entsprechenden
    Spalte machen.

    Neu: Der Kaffeedienst soll nach Reinigung der Kaffeemaschine
    bitte auch die {\color{red}Spülmaschine anstellen}.
    \begin{table}[h!]
        \Large
        \begin{tabular}{l | p{1.8cm} | p{1.8cm} | p{1.8cm} | p{1.8cm} | p{1.8cm}}
            % for begin
            
            \multicolumn{6}{l}{ {{ service.start_date.strftime('%d.%m.') }} -- {{ service.end_date.strftime('%d.%m.') }}} \\
            \hline
            \textbf{\hphantom{}{{ service.user.name }}} & Mo & Di & Mi & Do & Fr \\
            gespült (Zeit) &    &    &    &    &    \\
            gereinigt & \Huge\circ & \Huge\circ & \Huge\circ & \Huge\circ & \Huge\circ \\
            entkalkt & \Huge\circ & \Huge\circ & \Huge\circ & \Huge\circ & \Huge\circ \\
            \hline
            \multicolumn{6}{l}{}\\[3ex]
            
            % for end
        \end{tabular}
    \end{table}
\end{document}
